% Considera��es finais
\chapter{Considerações finais}
   
   Este trabalho apresenta o processo de desenvolvimento de um robô para solucionar o problema do cubo mágico utilizando componentes de baixo custo como a plataforma Arduíno. O desenvolvimento é composto por duas etapas, a construção de um protótipo e alocação de peças de hardware, e a programação destes componentes. Na parte de construção das peças, foi possível um entendimento melhor de como funciona os sistemas embarcados e outros componentes de hardware. Toda a parte de prototipação foi bem aplicada e gerou bons resultados para uma versão final. Com as peças e seus detalhes de tamanhos já definidos, foi possível a construção das peças na impressora 3D, isso permitiu que o projeto ganhasse vida. 
  
   
   Durante a parte de programação, foi possível colocar em prática conceitos e lógica de algoritmos e testes. Cada componente foi programado separadamente e após seu funcionamento completo, foi integrado com os outros. Como resultado do trabalho realizado, após vários testes de integração com os componentes, foi possível chegar a uma versão totalmente funcional capaz de completar o cubo de rubik. O objetivo definido no trabalho foi alcançado, com o robô produzido executando todo o processo de resolução em aproximadamente 13 minutos. No entanto, dependendo da configuração inicial do cubo, isso pode chegar a aproximadamente 18 minutos.
   
   
\section{Trabalhos Futuros}



    Uma das partes que demanda mais tempo é o envio das configurações do cubo para o robô, pois é preciso enviar todas as cores de cada peça e suas respectivas faces. Tendo isso em vista, como possíveis trabalhos futuros, pode-se apontar o desenvolvimento de um aplicativo para facilitar o envio de dados para o robô. Este aplicativo seria feito com a plataforma Android e com ele seria possível visualizar as configurações do cubo e enviar as cores de cada peça para o robô por meio de uma conexão bluetooth. 
   
   
   
   
