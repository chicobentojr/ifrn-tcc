% Resumo
\begin{center}
	{\Large{\textbf{Projeto e implementação de um robô que soluciona o problema do cubo de rubik usando Arduíno}}}
\end{center}

\vspace{1cm}

\begin{flushright}
	Autor: Helora Dana Cruz Monteiro\\
	Orientador(a): Dr. Plácido Antonio de Souza Neto
\end{flushright}

\vspace{1cm}

\begin{center}
	\Large{\textsc{\textbf{Resumo}}}
\end{center}

\noindent 
O cubo de Rubik, conhecido por muitos como cubo mágico, é um quebra cabeça tridimensional que possui uma série de passos para atingir uma solução. Por ser um quebra cabeça um tanto complexo, existem vários algoritmos compostos por sequências de passos para solucioná-lo. Mas para resolver o cubo de uma maneira eficaz e eficiente, podem-se utilizar mecanismos que facilitem o processo como, por exemplo, o uso da robótica. Robôs vêm sendo usados cada vez mais para execução de inúmeras tarefas, seja para indústria, prestação de serviços ou como forma de aprendizado nas escolas. O custo para projetar e construir um robô muitas vezes não é acessível pela variedade de componentes que ele possui, por isso sempre procuram-se meios fáceis e baratos de desenvolvê-los no âmbito escolar. Uma das propostas que vem sendo adotada para desenvolvimento de robôs é a plataforma arduíno, que vem como uma alternativa de baixo custo e de fácil aprendizado  onde só é necessário conhecimento em algoritmos para desenvolver uma aplicação. Tendo isso em vista, para uma melhor solução do problema do cubo e para melhorar o aprendizado de algoritmos foi desenvolvido um robô de baixo custo com a plataforma arduíno. Este trabalho tem como objetivo detalhar o processo de desenvolvimento do robô e seus resultados. 

\noindent\textit{Palavras-chave}: Cubo de rubik, Arduíno.