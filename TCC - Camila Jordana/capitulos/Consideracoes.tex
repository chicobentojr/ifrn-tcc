% Considera��es finais
\chapter{Considerações finais}

Diante do estudo apresentado, os objetivos definidos no início desse trabalho foram alcançados, sobretudo, pela realização da avaliação das particularidades das principais soluções do cubo de Rubik. A partir dessa análise mais aprofundada do cubo, foi possível conhecer os seus mecanismos, as relações matemáticas envolvendo a teoria dos grupos, conceitos e nomenclaturas específicas do quebra-cabeça.

A avaliação das soluções partiu de uma descrição detalhada do passo a passo de cada técnica, usando ilustrações para melhor explicitar o objetivo de cada etapa, detalhando a quantidade de casos contemplados por cada algoritmo e também comparando cada método para demonstrar as características de cada um.


A principal dificuldade encontrada no decorrer deste trabalho surgiu na etapa de implementação dos algoritmos. Devido à grande quantidade de casos contemplados pelo método de Fridrich, a codificação torna-se uma tarefa exaustiva.

A etapa de experimentação, detalhada no capítulo 4, foi de fundamental importância na observação do desempenho de cada algoritmo. A análise do tempo de execução mostrou que, em alguns casos, o método de Fridrich pode ter um desempenho inferior, apesar de apresentar uma otimização quando analisado o total de movimentos gerado na solução. O estudo de caso mostrado na seção 4.2 evidencia este fato.



\section{Trabalhos futuros}

A seguir, são listadas algumas sugestões para trabalhos posteriores, com foco em um estudo ainda mais detalhado. São elas:

\begin{itemize}

    \item Implementação do método de Petrus: apesar de não apresentar uma sequência de passos pré-definidas como os outros métodos e utilizar um conceito diferente das outras técnicas, é interessante apostar nessa solução, já que em alguns embaralhamentos pode apresentar um desempenho superior ao método de Fridrich. A comparação entre métodos avançados, como é o caso do Fridrich e do Petrus, é uma alternativa para encontrar uma solução ainda mais otimizada;
    
    
    \item Estudo de mais soluções: o estudo de outras soluções, inclusive menos conhecidas, leva a uma investigação mais aprofundado e permite avaliar, dentre várias alternativas, qual a melhor solução para uma determinada configuração do cubo.  
    
    \item Implementar um modelo de redução baseado em grafos: desenvolver uma representação da solução do cubo através de um modelo em grafo que descreva a sequência de movimentos necessários para a solução. 
    
\end{itemize}