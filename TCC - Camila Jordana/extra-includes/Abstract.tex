% Resumo em l�ngua estrangeira (em ingl�s Abstract, em espanhol Resumen, em franc�s R�sum�)
\begin{center}
	{\Large{\textbf{Implementation and Analysis of Algorithms for Solving the Rubik's Cube}}}
\end{center}

\vspace{1cm}

\begin{flushright}
	Author: Camila Jordana Ribeiro Teixeira\\
	Supervisor: Ph.D. Plácido Antônio de Souza Neto
\end{flushright}

\vspace{1cm}

\begin{center}
	\Large{\textsc{\textbf{Abstract}}}
\end{center}

\noindent The Rubik’s Cube is a game that uses seemingly random mechanisms to reach a solution. This game presents elements related to the universe of algorithmic solutions since it looks for solutions from sequences of steps. The most significant factor that contributes to the increase of its difficulty is the search for optimal solutions. Thus, the effort to find an optimal solution strengthens the learning of complex algorithmic solutions. Thus, this work presents an analysis of some methods that propose to solve the Rubik’s cube, to find an optimized solution. To achieve this goal, two algorithms were implemented: Fridrich and the traditional layered method. Afterwards, they were tested and analyzed, having as analysis parameters their respective execution times and the amount of movements resulting in each solution.

\noindent\textit{Keywords}: Rubik’s Cube, analysis of algorithms.