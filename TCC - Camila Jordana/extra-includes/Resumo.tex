% Resumo
\begin{center}
	{\Large{\textbf{Implementação e Análise de Algoritmos para Solução do Cubo de Rubik}}}
\end{center}

\vspace{1cm}

\begin{flushright}
	Autora: Camila Jordana Ribeiro Teixeira\\
	Orientador: Prof. Dr. Plácido Antônio de Souza Neto
\end{flushright}

\vspace{1cm}

\begin{center}
	\Large{\textsc{\textbf{Resumo}}}
\end{center}

\noindent O cubo de Rubik, também conhecido como cubo mágico, é um jogo da família dos quebra-cabeças que usa mecanismos, supostamente aleatórios, para atingir uma determinada solução. Esse jogo apresenta elementos relacionados ao universo de soluções algorítmicas, já que busca soluções a partir de sequências de passos. O fator mais significativo que contribui para o aumento da sua dificuldade é a busca por soluções ótimas. Assim, o empenho para encontrar um caminho considerado ótimo, aquele com um número reduzido de movimentos, fortalece ainda mais o aprendizado de soluções algorítmicas complexas. Tendo isso em vista, este trabalho apresenta uma análise de alguns métodos existentes que se propõem a solucionar o cubo de Rubik, a fim de encontrar uma solução otimizada. Para atingir esse objetivo é realizada a implementação de dois algoritmos, Fridrich e o método tradicional em camadas, para, em seguida, testá-los e analisá-los, tendo como parâmetros de análise seus respectivos tempos de execução e a quantidade de movimentos resultantes em cada solução.

\noindent\textit{Palavras-chave}: cubo de Rubik, análise de algoritmos.