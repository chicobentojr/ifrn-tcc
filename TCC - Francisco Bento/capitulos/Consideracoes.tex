% Considera��es finais
\chapter{Considerações finais}
\label{chap:consideracoes}

% <TODO> [DEPRECATED]: Remover o antigo capítulo 6 sobre os resultados e adicionar seu conteúdo com as considerações finais. Dizendo quais seriam os possíveis experimentos e como poderiam ser realizados. Na outra parte falar sobre as limitações e trabalhos futuros.

% As considerações finais formam a parte final (fechamento) do texto, sendo dito
% de forma resumida (1) o que foi desenvolvido no presente trabalho e quais os
% resultados do mesmo, (2) o que se pôde concluir após o desenvolvimento bem como
% as principais contribuições do trabalho, e (3) perspectivas para o
% desenvolvimento de trabalhos futuros, como listado nos exemplos de seção abaixo.
% O texto referente às considerações finais do autor deve salientar a extensão e
% os resultados da contribuição do trabalho e os argumentos utilizados estar
% baseados em dados comprovados e fundamentados nos resultados e na discussão do
% texto, contendo deduções lógicas correspondentes aos objetivos do trabalho,
% propostos inicialmente.

Neste trabalho nós propomos uma nova versão da plataforma GeoGuide \cite{omidvarTehrani2017} que leva em consideração os conceitos de IDR e a detecção de outliers para aprimorar a análise do usuário e as sugestões de novos pontos pela plataforma e com isso trazer mais opções sobre como o analista pode observar o dataset em busca de informações implícitas e quais os possíveis próximos passos para descobrir mais conhecimentos sobre aquele dataset em análise pelo usuário.

\section{Principais contribuições}

O foco deste trabalho é acrescentar novos fatores para a abordagem de orientação do GeoGuide: os IDRs e a detecção de outliers. Com isso, a preferência do usuário por determinadas regiões podem agora ser levada em consideração e também o processo de encontrar possíveis anomalias, e pontos que mereçam uma atenção mais específica para se compreender melhor, se tornou mais interativo e dinâmico, facilitando o uso por parte do usuário e dinamizando o processo iterativo de exploração do dataset.

\section{Limitações}

Uma limitação encontrada nessa proposta é o fato de que o cálculo das métricas base para o funcionamento do GeoGuide ainda é de uma complexidade muito alta e isso torna demorado o início da análise de um dataset e praticamente inviável para o analista no dia a dia conseguir analisar múltiplos datasets de grandes volumes.

\section{Trabalhos futuros}

Como trabalhos futuros nós propomos uma melhoria no processo para calcular as métricas base do GeoGuide, também podemos adicionar uma funcionalidade para o GeoGuide em que se leve em consideração múltiplos datasets e assim, com múltiplos contextos, nós podemos facilitar o processo para encontrar as razões pelas quais determinados outliers estão presentes naquele conjunto e assim dinamizar a exploração de vários datasets em um mesmo contexto. 