% Introdu��o
\chapter{Introdução}


In the last ten years, the search for terms such as big data, data analysis, and
data visualization has increased enormously. One of the reasons is that with the
advancement of technology and computers, we have been able to generate huge masses 
of data from different sources in various formats and in an incredibly small time.
Along with this came also new difficulties in the field of data analysis which is:
How to process these immense quantities quickly and efficiently?

Aiming at this problem, several types of research and tools have appeared that
try to solve or improve it in some way, either by proposing techniques to increase 
the performance of the analyzes or to perform the data cleaning or to improve the 
structure of how to save these data. Among these researches there is a part focused 
on how to visualize these large amounts of data and still more when it comes to 
spatial data, as it turns out to be a serious problem the more the amount of data 
grows, since the researcher could end up getting "lost "in the middle of so much 
information leaving their analysis greatly damaged.

In this context, one of these researches produced a new proposal that aims to 
improve the visualization and analysis of huge amounts of spatial data, the 
GeoGuide: a tool in which it is possible to load a generic dataset with spatial 
data (latitude and longitude attributes) and metadata and then visualize it on a 
global map to better navigate between them. Along with this there is also the 
concept of diversity and similarity that serves for an approach in which the
researcher expands its area of research through highlights of similar points in 
distinct areas of a single point chosen by him. for example: Joana is a culinary
enthusiast and wants to find new restaurants in neighborhoods that serve Brazilian
food at a price range from $20 to $ 100. In a few clicks, you can find these 
suggestions in GeoGuide.

However, a new problem arises that is the availability of Joana to be able to 
reach certain neighborhoods because she does not have a car and needs public 
transportation to transit in her city. Aiming at this new feature, GeoGuide is 
adding the concepts of regions of interest (neighborhoods, in the case of Joana) 
so that the researcher can, implicitly (using the mouse movement), demonstrate 
which region is more interesting for him and thus avoid one more step that would 
be the process to exclude the suggestions of the GeoGuide that would be in unavailable
places for Joana to access. With this concept it is also possible to solve another
problem that would be the case of Marcos, who is passionate about travel and wants
to redo a trip to Italy, however, he decided that he does not want to visit the 
same sights. So to avoid a new process, Mark would get suggestions outside his 
region of interest (which would be his last places visited in Italy), and then 
enjoy his trip.


\section{Objetivos}

Nesta seção são definidos os objetivos gerais e específicos do trabalho.

\subsection{Objetivos Gerais}

\begin{itemize}
  \item Objetivo geral 1\ldots
  \item Objetivo geral 2\ldots  
\end{itemize}

\subsection{Objetivos Específicos} 

Lista de objetivos específicos do trabalho\ldots

\begin{itemize}
  \item Objetivo específico 1\ldots
  \item Objetivo específico 2\ldots  
\end{itemize}

\section{Metodologia}

Na metodologia é descrito o método de investigação e pesquisa para o
desenvolvimento e implementação do trabalho que está sendo proposto.

\section{Organização do trabalho}

Nesta seção deve ser apresentado como está organizado o trabalho, sendo
descrito, portanto, do que trata cada capítulo.
