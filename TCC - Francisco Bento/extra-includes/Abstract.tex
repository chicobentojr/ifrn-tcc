% Resumo em l�ngua estrangeira (em ingl�s Abstract, em espanhol Resumen, em franc�s R�sum�)
\begin{center}
	{\Large{\textbf{\thesistitle}}}
\end{center}

\vspace{1cm}

\begin{flushright}
	Author: \thesisauthor\\
	Supervisor: \thesissupervisor
\end{flushright}

\vspace{1cm}

\begin{center}
	\Large{\textsc{\textbf{Abstract}}}
\end{center}

\noindent The volume of spatial data generated has grown daily. These characteristics complicate the data analysts investigation. Besides that, these large volumes may contain data that are discrepant and hard to analyze. In this paper it will be presented a proposal of a new version of GeoGuide, a framework that aims assist the analyst during his exploration of large volumes of spatial data, taking into account the regions of user interest and the outlier data that are present in the set. It will also be demonstrated the operation of this approach and its advantages based on real data examples.

\noindent\textit{Keywords}: Spatial data, User interest, Outliers.