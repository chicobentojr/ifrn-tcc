% Resumo
\begin{center}
	{\Large{\textbf{\thesistitle}}}
\end{center}

\vspace{1cm}

\begin{flushright}
	Autor: \thesisauthor\\
	Orientador(a): \thesissupervisor
\end{flushright}

\vspace{1cm}

\begin{center}
	\Large{\textsc{\textbf{Resumo}}}
\end{center}

TODO: Fazer Resumo

\noindent O resumo deve apresentar de forma concisa os pontos relevantes de um
texto, fornecendo uma visão rápida e clara do conteúdo e das conclusões do
trabalho. O texto, redigido na forma impessoal do verbo, é constituído de uma
sequência de frases concisas e objetivas e não de uma simples enumeração de
tópicos, não ultrapassando 500 palavras, seguido, logo abaixo, das palavras
representativas do conteúdo do trabalho, isto é, palavras-chave e/ou
descritores. Por fim, deve-se evitar, na redação do resumo, o uso de parágrafos
(em geral resumos são escritos em parágrafo único), bem como de fórmulas,
diagramas e símbolos, optando-se, quando necessário, pela transcrição na forma
extensa, além de não incluir citações bibliográficas.

\noindent\textit{Palavras-chave}: Palavra-chave 1, Palavra-chave 2, Palavra-chave 3.